\documentclass[a4paper, top=10mm]{article}
%for writing from the top
\usepackage{fullpage}
%for math
\usepackage{amsmath}
\usepackage{mathrsfs}
\usepackage{amsthm}
\usepackage{amsfonts}
%for images
\usepackage{graphicx}
%for color
\usepackage{xcolor}
%for title
\title{\textbf{\huge{NCBC Rowers}}}
\author{Enigma n\textsuperscript{o}5}
\date{2\textsuperscript{nd} December 2022}

\newtheorem*{hint}{Hint}

\addtolength{\voffset}{-2cm}
\addtolength{\textheight}{5cm}


\begin{document}
	\maketitle
	
	New College Boat Club (NCBC) rowers are extremely regular (we can therefore consider their speed to be constant).
	
	They go up in the river, and reach a buoy while crossing a bridge.
	After 15 min, they turn around (now going down the river).
	When they reach the buoy again, it has drifted 1km below the bridge.
	
	What is the speed of the river's flow?
	
	\vspace{1cm}
	
	\begin{center}
		\includegraphics[height=200pt]{05NCBC_crest.png}\\
		New College Boat Club Crest.
	\end{center}
	
	\vspace{3cm}
	
	\textbf{The answer is the current speed in m/h (rounded to the closes integer if needed).}
	
\end{document}